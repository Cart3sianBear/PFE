\documentclass{article}

\usepackage[utf8]{inputenc}
\usepackage{graphicx}
\usepackage{amsmath}
\usepackage{lscape}
\usepackage{braket}
\usepackage{multirow}
\usepackage{hyperref}
\usepackage{listings}
\usepackage{color}

\usepackage{fancyhdr}
\usepackage[headheight = 2cm,bottom=3.5cm]{geometry}
\pagestyle{fancy}
\renewcommand{\headrulewidth}{1pt}
\fancyhead[L]{\includegraphics[height=1.5cm]{logoe2m.jpg}} 
\fancyhead[C]{\textsc{Mechanical conception and geometrical optimization\\of a car carbon fiber bodywork\\}}
\fancyhead[R]{}
\renewcommand{\footrulewidth}{1pt}
\fancyfoot[C]{} 
\fancyfoot[L]{Simon Froelicher\\ Baptiste Legouix\\ Tutor : Cedric Laurent}
\fancyfoot[R]{\thepage}

\begin{document}
\title{\textsc{PFE report} \\
\vspace{10pt}\hspace*{-30pt}\begin{tabular}{cc}
\multirow{4}{*}{
\begin{minipage}{0.25\textwidth}\vspace{15pt}\includegraphics[scale=0.53]{logoensem.png}\end{minipage}\hspace{-10pt}} & \rule{0.8\linewidth}{0.5pt}\vspace{10pt} \\ & Mechanical conception and geometrical optimization \\ &  of a car carbon fiber bodywork \\ & \rule{0.8\linewidth}{2pt}
\end{tabular}}
\author{Simon Froelicher\\ Baptiste Legouix\\ Tutor : Cedric Laurent}
%\date{\today\\[30pt] \huge \ccbynd}
\date{\today}

\clearpage
\maketitle
\thispagestyle{empty}

\tableofcontents
\newpage

\setcounter{page}{1}

\section{Solid mechanical modeling}

\subsection{Theory of composite materials}

Fondamentaux de MMC + relations mathématiques qui permettent de caractériser les matériaux carbone (monolithique et sandwich), tenant compte de leurs symétries.

+ Etude de résistance du matériau \textcolor{red}{(à éclaircir)}.

\hrulefill

\subsubsection{Continuum mechanics}

La dynamique locale d'un milieu continu est décrite en mécanique par la relation suivante, assimilable à la seconde loi de Newton :

\[
\vec{f}=\rho\frac{\partial^2 \vec{u}}{\partial t^2}
\]

Où $\vec{f}$ est la force volumique appliquée sur un élément infinitésimal du milieu et $\vec{u}$ son déplacement par rapport à l'état initial considéré (généralement l'état où chaque élément est dans un référentiel inertiel, c'est-à-dire avant qu'on n'envisage l'application des forces).

Le problème de cette représentation est qu'elle n'est pas complètement locale, au sens où $\vec{u}$ n'est pas une grandeur infinitésimale (au contraire de $\vec{f}$. Il faudra y remédier, mais gardons ça pour plus tard.

La continuité du milieu est assurée par des forces que s'appliquent mutuellement les éléments. Ces forces (volumiques) sont d'une autre nature que les éventuels forces extérieures $\vec{f}_s$ ($s$ pour\emph{source}), et s'écrivent comme la divergence d'un tenseur de rang deux $\sigma$ appellé \emph{tenseur des contraintes de Cauchy} :

\[
\nabla\cdot\underline{\underline{\sigma}}+\vec{f}_s=\rho\frac{\partial^2 \vec{u}}{\partial t^2}
\]

Montrons que la conservation du moment cinétique implique la symétrie de ce tenseur.

On écrit d'abord (sans démonstration) un bilan de moment cinétique sur un domaine quelconque :

\[
\frac{d}{dt}\iiint\vec{x}\times\rho\frac{\partial \vec{u}}{\partial t}\; dV=\iiint\vec{x}\times\vec{f}_s \; dV+\iint\vec{x}\times\underline{\underline{\sigma}}\; \vec{dS}
\]

Si l'on souhaite accéder à une version locale, il nous faut transformer l'intégrale de surface en intégrale de volume, dans l'idée :

\[
\frac{d}{dt}\iiint\vec{x}\times\rho\frac{\partial \vec{u}}{\partial t}\; dV=\iiint\vec{x}\times\vec{f}_s \; dV+\iiint\nabla\cdot(\vec{x}\times\underline{\underline{\sigma}})\; dV
\]

Néanmoins, le produit vectoriel d'un vecteur par une matrice n'est pas conventionnel à ma connaissance. On peut cependant définir ce dernier comme :

\[
\vec{v}\times \underline{\underline{M}} = \varepsilon_{ikl}v_kM_{jl}
\]

\textcolor{red}{(distinction indices covariants/contravariants ?)}

Où est noté $\varepsilon$ le \emph{symbole de Levi-Civita}. En adoptant cette définition, la décomposition suivante devient valide :

\[
\iiint\nabla\cdot(\vec{x}\times\underline{\underline{\sigma}})\; \vec{dV} = \iiint\nabla\vec{x}\times\underline{\underline{\sigma}}\; \vec{dV}+\iiint\vec{x}\times\nabla\cdot\underline{\underline{\sigma}}\; \vec{dV}
\]

\[
\iiint\nabla\cdot(\vec{x}\times\underline{\underline{\sigma}})\; \vec{dV} = \iiint\underline{\underline{I}}\times\underline{\underline{\sigma}}\; \vec{dV}+\iiint\vec{x}\times\nabla\cdot\underline{\underline{\sigma}}\; \vec{dV}
\]

Où $\underline{\underline{I}}\times\underline{\underline{\sigma}}$ n'a de sens que si l'on définit :

\[
\underline{\underline{M}}\times\underline{\underline{N}}=\epsilon_{ikl}M_{kj}N_{jl}
\]

On vérifie la décomposition proposée en écrivant explicitement :

\[
\frac{\partial}{\partial x_j}(\varepsilon_{ikl}x_k\sigma_{jl}) = \varepsilon_{ikl}\frac{\partial x_k}{\partial x_j}\sigma_{jl}+ \varepsilon_{ikl}x_k\frac{\partial\sigma_{jl}}{\partial x_j}
\]

On trouve finalement (hypothèse du fluide incompressible appliquée) :

\[
\iiint\left(\vec{x}\times\rho\frac{\partial^2 \vec{u}}{\partial t^2}-\vec{x}\times\vec{f}_s -\underline{\underline{I}}\times\underline{\underline{\sigma}}-\vec{x}\times\nabla\cdot\underline{\underline{\sigma}}\right)\; dV=0
\]

Ceci étant vrai $\forall V$, on abouti à la relation locale :

\[
\vec{x}\times\left(\rho\frac{\partial^2 \vec{u}}{\partial t^2}-\vec{f}_s -\nabla\cdot\underline{\underline{\sigma}}\right)-\underline{\underline{I}}\times\underline{\underline{\sigma}}=0
\]

Mais d'après la relation dont nous sommes partis, on remarque que la quantité dans les parenthèses s'annule ! Autrement dit :

\[
\underline{\underline{I}}\times\underline{\underline{\sigma}}=0
\]

Ou encore :

\[
\varepsilon_{ikl}\delta_{kj}\sigma_{jl} = 0
\]

\[
\varepsilon_{ikl}\sigma_{kl} = 0
\]

Connaissant la structure algébrique de $\varepsilon$, il est inévitable que :

\[
\sigma_{ij} = \sigma_{ji}
\]

La question de la modélisation locale des efforts structurels par le tenseur des contraintes étant traitée, revenons maintenant à la transformation :

\[
\vec{x} \rightarrow \vec{x'} = \vec{x}+\vec{u}
\]

Pour écrire la déformation subie par un élément :

\[
dx'_i = \frac{\partial x'_i}{\partial x_j}dx_j
\]

\[
dx'_i = \left(\frac{\partial x_i}{\partial x_j}+\frac{\partial u_i}{\partial x_j}\right)dx_j
\]

\[
dx'_i = \left(\delta_{ij}+\frac{\partial u_i}{\partial x_j}\right)dx_j
\]

A ce stade vient naturellement l'envie de considérer le tenseur de rang deux $\underline{\underline{F}}=\underline{\underline{I}}+\nabla u$ comme objet caractérisant une déformation, et de définir l'élasticité sur la base de celui-ci. Pourtant, nous préfèrerons utiliser \textcolor{red}{(pourquoi ?)} le tenseur suivant, que nous qualifierons de \emph{tenseur des déformations} :

\[
\underline{\underline{\varepsilon}}=\underline{\underline{F}}^\top\underline{\underline{F}}-\underline{\underline{I}}
\]

Manifestement symétrique, et autrement écrit :

\[
\underline{\underline{\varepsilon}} = (\nabla u)^\top + \nabla u + (\nabla u)^\top\nabla u
\]

De façon à ce que :

\[
\vec{dx}^\top \underline{\underline{\varepsilon}} \; \vec{dx} = \vec{dx}^\top (\underline{\underline{F}}^\top\underline{\underline{F}}-\underline{\underline{I}}) \; \vec{dx}
\]

\[
\vec{dx}^\top \underline{\underline{\varepsilon}} \; \vec{dx} = ||\vec{dx'}||^2 -||\vec{dx}||^2
\]

Nous avons les deux tenseurs qu'il nous faut pour pouvoir introduire l'objet qui caractérisera finalement le matériau qu'on voudra considérer.

Un solide sera décrit en mécanique par sa \emph{loi de comportement} qui lit $\sigma$ à $\epsilon$. Dans tout ce rapport, nous n'emploierons que la \emph{loi de Hooke} qui s'applique aux régimes linéaires :

\[
\underline{\underline{\sigma}}=\underline{\underline{\underline{\underline{C}}}}\;
\underline{\underline{\varepsilon}}
\]

Le tenseur $\underline{\underline{\underline{\underline{C}}}}$ est de rang quatre, mais du fait qu'il conserve la symétrie du tenseur de rang deux qu'on lui donne, toute l'information qu'il contient peut-être contenue dans une matrice $6\times 6$ :

\[
\left(\begin{matrix}
\sigma_{11} \\ \sigma_{22} \\ \sigma_{33} \\ \sigma_{23} \\ \sigma_{13} \\ \sigma_{12} \\\end{matrix}\right) = \underline{\underline{C}}
\left(\begin{matrix}
\varepsilon_{11} \\ \varepsilon_{22} \\ \varepsilon_{33} \\ \varepsilon_{23} \\ \varepsilon_{13} \\ \varepsilon_{12} \\\end{matrix}\right)
\]

On appelle cette réduction de l'ordre du problème le passage à la \emph{notation de Voigt}.

Cette matrice $6\times 6$ caractérise donc entièrement un matériau, en toute généralité (à ceci près qu'on se restreint au domaine linéaire).

\subsubsection{Monolitic composite}

La désignation \emph{composite monolitique} s'oppose à \emph{composite stratifié}, ce dernier étant constitué de couches différemments orientées.

La matrice de raideur d'un tel composite s'écrit dans son repère principal (base propre du tenseur des contraintes) :

\[
\underline{\underline{C}}=\left(\begin{matrix}
C_{11} & C_{12} & C_{13} & 0 & 0 & 0 \\
C_{12} & C_{22} & C_{23} & 0 & 0 & 0 \\
C_{13} & C_{23} & C_{33} & 0 & 0 & 0 \\
0 & 0 & 0 & C_{44} & 0 & 0 \\
0 & 0 & 0 & 0 & C_{55} & 0 \\
0 & 0 & 0 & 0 & 0 & C_{66}
\end{matrix}\right)
\]

Ces neuf coefficients sont en bijection avec les grandeurs estimées lors d'essais visants à caractériser le matériau : les trois modules d'Young, les trois coefficients de Poisson et les trois modules de cisaillement.

En dehors des directions principales, on doit effectuer une transformation linéaire pour trouver \textcolor{red}{(démonstration)} :

\[
\underline{\underline{C'}}=\left(\begin{matrix}
C'_{11} & C'_{12} & C'_{13} & 0 & 0 & C'_{16} \\
C'_{12} & C'_{22} & C'_{23} & 0 & 0 & C'_{26} \\
C'_{13} & C'_{23} & C'_{33} & 0 & 0 & C'_{36} \\
0 & 0 & 0 & C'_{44} & 0 & 0 \\
0 & 0 & 0 & 0 & C'_{55} & 0 \\
C'_{16} & C'_{26} & C'_{36} & 0 & 0 & C'_{66}
\end{matrix}\right)
\]

Avec des relations algébriques entre les coefficients précédents et ceux-ci, aucun degré de liberté excédentaire n'étant en fait introduit. 

\subsubsection{Sandwitch-structured composite}

\subsubsection{Laminated composite}

\subsection{Determination of stiffness matrices}

Essais mécaniques et relation entre les mesures et les coefficients de la matrice de raideur.

+ Essais de rupture.

\subsection{Finite element resolution}

\subsubsection{Computational solid mechanics}

Optionnel

\subsubsection{Static analysis}

Etude statique sous AnSys.

\subsubsection{Transcient analysis}

Etude fréquentielle ou transitoire, sur la base de mesures accélérométriques \textcolor{red}{(non-réalisées, donc à estimer ?)}.

\section{Aerodynamic modeling}

\subsection{Theory of fluid dynamics}

Fondamentaux

\subsection{Finite element resolution}

\subsubsection{Computational fluid dynamics}

Optionnel

\subsubsection{Analysis of laminar flow}

Détermination du $C_x$ et $C_z$ sous AnSys.

\subsubsection{Analysis of turbulent flow}

Idem

\section{Composite bodywork design}

\subsection{Specifications}

Lister les contraintes imposées par les règles

\subsection{Complete modeling based on finite element results}

Comment estimer les pertes à partir des deux simulations (solide et fluide) ? En les effectuant indépendamment puis en mettant des poids sur leurs pertes respectives, ou en effectuant le calcul fluide suivi du calcul solide prenant en compte les pressions calculées ? 

\subsection{Design choices}

\section{Geometrical optimization}

\end{document}

\documentclass{article}

\usepackage[utf8]{inputenc}
\usepackage{graphicx}
\usepackage{amsmath}
\usepackage{lscape}
\usepackage{braket}
\usepackage{multirow}
\usepackage{hyperref}
\usepackage{listings}
\usepackage{color}

\usepackage{fancyhdr}
\usepackage[headheight = 2cm,bottom=3.5cm]{geometry}
\pagestyle{fancy}
\renewcommand{\headrulewidth}{1pt}
\fancyhead[L]{\includegraphics[height=1.5cm]{logoe2m.jpg}} 
\fancyhead[C]{\textsc{Mechanical conception and geometrical optimization\\of a car carbon fiber bodywork\\}}
\fancyhead[R]{}
\renewcommand{\footrulewidth}{1pt}
\fancyfoot[C]{} 
\fancyfoot[L]{Simon Froelicher\\ Baptiste Legouix\\ Tutor : Cedric Laurent}
\fancyfoot[R]{\thepage}

\begin{document}
\title{\textsc{PFE report} \\
\vspace{10pt}\hspace*{-30pt}\begin{tabular}{cc}
\multirow{4}{*}{
\begin{minipage}{0.25\textwidth}\vspace{15pt}\includegraphics[scale=0.53]{logoensem.png}\end{minipage}\hspace{-10pt}} & \rule{0.8\linewidth}{0.5pt}\vspace{10pt} \\ & Mechanical conception and geometrical optimization \\ &  of a car carbon fiber bodywork \\ & \rule{0.8\linewidth}{2pt}
\end{tabular}}
\author{Simon Froelicher\\ Baptiste Legouix\\ Tutor : Cedric Laurent}
%\date{\today\\[30pt] \huge \ccbynd}
\date{\today}

\clearpage
\maketitle
\thispagestyle{empty}

\tableofcontents
\newpage

\setcounter{page}{1}

\section{Solid mechanical modeling}

\subsection{Theory of composite materials}

Fondamentaux de MMC + relations mathématiques qui permettent de caractériser les matériaux carbone (monolithique et sandwich), tenant compte de leurs symétries.

+ Etude de résistance du matériau \textcolor{red}{(à éclaircir)}.

\subsection{Determination of stiffness matrices}

Essais mécaniques et relation entre les mesures et les coefficients de la matrice de raideur.

+ Essais de rupture.

\subsection{Finite element resolution}

\subsubsection{Computational solid mechanics}

Optionnel

\subsubsection{Static analysis}

Etude statique sous AnSys.

\subsubsection{Transcient analysis}

Etude fréquentielle ou transitoire, sur la base de mesures accélérométriques \textcolor{red}{(non-réalisées, donc à estimer ?)}.

\section{Aerodynamic modeling}

\subsection{Theory of fluid dynamics}

Fondamentaux

\subsection{Finite element resolution}

\subsubsection{Computational fluid dynamics}

Optionnel

\subsubsection{Analysis of laminar flow}

Détermination du $C_x$ et $C_z$ sous AnSys.

\subsubsection{Analysis of turbulent flow}

Idem

\section{Composite bodywork design}

\subsection{Specifications}

Lister les contraintes imposées par les règles

\subsection{Complete modeling based on finite element results}

Comment estimer les pertes à partir des deux simulations (solide et fluide) ? En les effectuant indépendamment puis en mettant des poids sur leurs pertes respectives, ou en effectuant le calcul fluide suivi du calcul solide prenant en compte les pressions calculées ? 

\subsection{Design choices}

\section{Geometrical optimization}

\end{document}
